\documentclass[12pt]{beamer}

\usepackage[frenchb]{babel}
\usepackage[T1]{fontenc}
\usepackage[utf8x]{inputenc}    % accent
\usepackage{graphicx}           % pictures
\usepackage{multimedia}         % sounds and movies

\usetheme{Warsaw}





%%%%%%%%%%%%%%%%%
% PREAMBULE     %
%%%%%%%%%%%%%%%%%
\title{Soutenance de projet}
\subtitle{Picross}
\author{Groupe B}
\date{\today}
\institute{Université du Maine}
%%%%%%%%%%%%%%%%%
%%%%%%%%%%%%%%%%%
\setbeamertemplate{itemize item}[circle]
\hypersetup{
        pdfpagemode = FullScreen,% afficher le pdf en plein écran
        pdfauthor   = {groupe B},%
        pdftitle    = {Soutenance}%
        pdfsubject  = {Soutenance},%
        pdfcreator  = {PDFLaTeX},%
}
\setbeamertemplate{navigation symbols}{
        \insertslidenavigationsymbol
        \insertframenavigationsymbol
        %\insertsubsectionnavigationsymbol
        %\insertsectionnavigationsymbol
        %\insertdocnavigationsymbol
        %\insertbackfindforwardnavigationsymbol
}
% Faire apparaître un sommaire avant chaque section
\AtBeginSection[]{
        \begin{frame}
        \begin{center}{\Large Plan }\end{center}
        %%% affiche en début de chaque section, les noms de sections et
        %%% noms de sous-sections de la section en cours.
        \tableofcontents[currentsection,hideothersubsections]
        \end{frame} 
}







%%%%%%%%%%%%%%%%%
% BEGINNING     %
%%%%%%%%%%%%%%%%%
\begin{document}
        \frame{\titlepage}
        \frame{\tableofcontents}




% % % % % % %
% FRAME     %
% % % % % % %
\section[]{Première section}
\begin{frame}
       \frametitle{Le Pingouin}
        \framesubtitle{un ennemi mortel du Batman}

        \begin{columns}[c] % column structure
        	\begin{column}{5.5cm} % right column 
			\begin{itemize}
				\item bois ou métal;
				\bigskip
				\item hanche double;
				%\bigskip
				\item 50 cm de long;
			\end{itemize}

        	\end{column}
        	\begin{column}{5.0cm} % left column
                        %\includegraphics[scale=0.3]{data/Shehnai.jpg}
			mortadelle == ventre.getContent()
        	\end{column}
        \end{columns}

% END FRAME 
\end{frame}


                
% % % % % % %
% FRAME     %
% % % % % % %
\begin{frame}
        \frametitle{Pour le musicien}
        \framesubtitle{cookbook du Shenaï en 1 diapo}

        \begin{columns}[c] % column structure
        	\begin{column}{5.5cm} % right column 
			\begin{exampleblock}{Instrument seul}
				\begin{itemize}
					\item s'étend sur 2 octaves;
					\item souffle continu très utilisé;
					\item instrument solo depuis récemment;
				\end{itemize}
			\end{exampleblock}
        	\end{column}
        	\begin{column}{5.0cm} % left column
			\begin{exampleblock}{En groupe}
        			\begin{itemize}
					\item souvent joué en groupe, ou \textit{naubat};
					\item accompagné d'un \textit{khurdak};
        			\end{itemize}
			\end{exampleblock}
        	\end{column}
        \end{columns}

% END FRAME 
\end{frame}



                
% % % % % % %
% FRAME     %
% % % % % % %
\section[]{Basmillah Khan}
\begin{frame}
        \frametitle{Sa vie}
        \framesubtitle{21 mars 1913 - 21 août 2006}

        \begin{columns}[c] % column structure
        	\begin{column}{5.5cm} % right column 
        	\end{column}
        	\begin{column}{5.0cm} % left column
        		\begin{itemize}
				\item né à Dumraon, dans le Bihar;
		\bigskip
				\item famille de musicien, notamment \textit{Rasool Bux Khan} son grand-père;
		\bigskip
				\item mort à Bénarès;
        		\end{itemize}
        	\end{column}
        \end{columns}

		\bigskip
        \begin{center}
        	\begin{quote}
			Music has no caste -- Basmillah Khan
        	\end{quote}
        \end{center}

% END FRAME 
\end{frame}


                

% % % % % % %
% FRAME     %
% % % % % % %
\begin{frame}
        \frametitle{Thaat : Todi}
        \framesubtitle{}
	Un thaat du matin contenant 5 ragas\\
	\bigskip
        \begin{tabular*}{0.75\textwidth}{ c | c | c | c | c | c | c | c}
		Do & Re$\flat$ & Mi$\flat$ & Fa\# & Sol & La$\flat$ & Si & Do\\
		S & r & g & M & P & d & N & S'
        \end{tabular*}
	
% END FRAME 
\end{frame}




\end{document}
