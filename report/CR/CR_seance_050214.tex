%%%%%%%%%%%%%%%%%%%%%%%%%%%%%%%%%
% INCLUSION PREAMBULE COMMUN 	%
%%%%%%%%%%%%%%%%%%%%%%%%%%%%%%%%%
% définition et packages du présent fichier
%%%%%%%%%%%%%%%%%%%%%%%%%
% PACKAGES              %
%%%%%%%%%%%%%%%%%%%%%%%%%
\documentclass{report}
\usepackage[utf8x]{inputenc}  % accents
\usepackage{geometry}         % marges
\usepackage[francais]{babel}  % langue
\usepackage{graphicx}         % images
\usepackage{verbatim}         % texte préformaté
\usepackage{fancyhdr}         % fancy






% Titre de ce fichier
\newcommand{\titre}{Compte-rendu de séance}
\newcommand{\titrehead}{Compte-rendu de séance}
% Inclusion du préambule commun
%%%%%%%%%%%%%%%%%%%%%%%%%
% PRÉAMBULE             %
%%%%%%%%%%%%%%%%%%%%%%%%%
\title{\titre{}}
\author{}
% laisser vide pour date de compilation
\date{} 

% FORMAT PAGES         
\pagestyle{fancy} % nom du rendu (définit les lignes suivantes)
        \lhead{} % left head
        \chead{\titrehead{}} % center head
        \rhead{} % right head
        \lfoot{} % left foot
        \cfoot{\thepage} % center foot
        \rfoot{} % right foot


% Ce fichier est un préambule commun à toutes les sources LaTeX.
% Il est inclus par toutes les sources et permet d'avoir un formatage commun facilement modifiable.









%%%%%%%%%%%%%%%%%%%%%%%%%
% BEGIN                 %
%%%%%%%%%%%%%%%%%%%%%%%%%
\begin{document}




\chapter*{Rapport de séance du Groupe B}	
\section*{05 Février 2014}
    	\paragraph*{}
		L'objectif principal de la séance était de commencer le cahier des charges, en analysant les besoins et les moyens.
		Une ébauche a été écrite pendant la séance, puis de premiers choix sur l'interface graphique et les structure de données de base ont été faits.
    	\paragraph*{}
		Avec un Etherpad, l'ensemble du groupe a pu travailler au squelette du cahier des charges. Le remplissage s'est fait au fur et à mesure de la séance.
		Les membres du groupe rédigeront le cahier des charges chacun de leur côté depuis l'Etherpad pour la prochaine séance.


\section*{Signatures}

    	\begin{tabular*}{0.75\textwidth}{c | c | c | c | c | c}
    	    Bourneuf Lucas & Charlie Maréchal & Ewen Cousin & Nicolas Bourdin & Julien Le Gall & Jaweed Parwany\\
     	     & & & & &
    	\end{tabular*}
\end{document}
% END



