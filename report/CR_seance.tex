%%%%%%%%%%%%%%%%%%%%%%%%%%%%%%%%%
% INCLUSION PREAMBULE COMMUN 	%
%%%%%%%%%%%%%%%%%%%%%%%%%%%%%%%%%
% définition et packages du présent fichier
%%%%%%%%%%%%%%%%%%%%%%%%%
% PACKAGES              %
%%%%%%%%%%%%%%%%%%%%%%%%%
\documentclass{report}
\usepackage[utf8x]{inputenc}  % accents
\usepackage{geometry}         % marges
\usepackage[francais]{babel}  % langue
\usepackage{graphicx}         % images
\usepackage{verbatim}         % texte préformaté
\usepackage{fancyhdr}         % fancy






% Titre de ce fichier
\newcommand{\titre}{Compte-rendu de séance}
% Inclusion du préambule commun
%%%%%%%%%%%%%%%%%%%%%%%%%
% PRÉAMBULE             %
%%%%%%%%%%%%%%%%%%%%%%%%%
\title{\titre{}}
\author{}
% laisser vide pour date de compilation
\date{} 

% FORMAT PAGES         
\pagestyle{fancy} % nom du rendu (définit les lignes suivantes)
        \lhead{} % left head
        \chead{\titrehead{}} % center head
        \rhead{} % right head
        \lfoot{} % left foot
        \cfoot{\thepage} % center foot
        \rfoot{} % right foot


% Ce fichier est un préambule commun à toutes les sources LaTeX.
% Il est inclus par toutes les sources et permet d'avoir un formatage commun facilement modifiable.









%%%%%%%%%%%%%%%%%%%%%%%%%
% BEGIN                 %
%%%%%%%%%%%%%%%%%%%%%%%%%
\begin{document}




\section{Séance du 24 Janvier 2014}
    	\paragraph*{}
	Les rôles principaux ont été attribués :
	\begin{description}
		\item[Chef de projet : ] Lucas BOURNEUF;
		\item[Documentaliste : ] Charlie Maréchal;
	\end{description}
    	\paragraph*{}
	Des discussions ont eu lieu sur les interfaces graphiques, le principe de l'aide et de la persistance des données.
	Les structures de données employées ont été définie comme ressources critiques, nécessaires à la définition des autres parties du projet. 
	De premières idées ont été émises sur ces structures, et sur les différentes parties du projet.
    	\paragraph*{}
	Principales parties du projet :
	\begin{itemize}
		\item[- structures de données;]
		\item[- GUI;]
		\item[- moteur de jeu;]
		\item[- aide;]
		\item[- documentation;]
	\end{itemize}
    	\paragraph*{}
	Technologies utilisées :
	\begin{itemize}
		\item[- github;]
		\item[- GTK2, glade et vr,] pour la réalisation de la GUI;
		\item[- ruby 1.9;]
		\item[- Marshall/YamL/ORM,] selon si les données doivent être lisibles ou non;
		\item[- ganttproject,] pour la réalisation et la sauvegarde des diagramme de gantt;
		\item[- rdoc] pour la doc;
		\item[- \LaTeX,] pour les rapport;
	\end{itemize}
    	\paragraph*{}
	De premières questions à poser aux clients ont été définies, notamment concernant le principe de l'aide et la disposition et le contenu de la GUI.




\section{Séance du 31 Janvier 2014}
    	\paragraph*{}
\section{Signatures}
    	\begin{tabular*}{0.75\textwidth}{c | c | c | c | c | c}
    	    Bourneuf Lucas & Charlie Maréchal & Ewen Cousin & Nicolas Bourdin & Julien Le Gall & Jaweed Parwany\\
     	     & & & & &
    	\end{tabular*}
\end{document}
% END
